\documentclass[12pt]{article} 
% Formatting
\tolerance=1000
\usepackage[margin=1.2in]{geometry}

\input{../custom.tex}

% Packages

% \usepackage{amssymb,latexsym}
\usepackage{amssymb,amsfonts,amsmath,latexsym,amsthm}
\usepackage[usenames,dvipsnames]{color}
\usepackage[]{graphicx}
\usepackage[space]{grffile}
\usepackage{mathrsfs}   % fancy math font
% \usepackage[font=small,skip=0pt]{caption}
\usepackage[skip=0pt]{caption}
\usepackage{subcaption}
\usepackage{verbatim}
\usepackage{url}
\usepackage{bm}
\usepackage{dsfont}
\usepackage{extarrows}
\usepackage{multirow}
% \usepackage{wrapfig}
% \usepackage{epstopdf}
\usepackage{rotating}
\usepackage{tikz}
\usetikzlibrary{fit}					% fitting shapes to coordinates
%\usetikzlibrary{backgrounds}	% drawing the background after the foreground


% \usepackage[dvipdfm,colorlinks,citecolor=blue,linkcolor=blue,urlcolor=blue]{hyperref}
\usepackage[colorlinks,citecolor=blue,linkcolor=blue,urlcolor=blue]{hyperref}
%\usepackage{hyperref}
\usepackage[authoryear,round]{natbib}



\title{Homework on graphical models}
%\author{Jeff Miller}
\date{}




\begin{document}
\maketitle

\begin{enumerate}
    \item\label{factor} Show that if $p(x,y,z) = g(x,z) h(y,z)$ for some nonnegative functions $g$ and $h$, then $p(x,y|z) = p(x|z)p(y|z)$ (i.e., $X\perp Y\mid Z$).
\item In this exercise, you will derive the ``separation criterion'' for conditional independence in undirected graphical models.
    Suppose $p(x_1,\ldots,x_n)$ respects an undirected graph $G$.
    Suppose $A,B,C$ are disjoint subsets of vertices of $G$, such that all paths from $A$ to $B$ are blocked by $C$.
    \begin{enumerate}
        \item Divide the set of vertices not in $A$, $B$, or $C$ into two subsets: let $D$ be the subset that can be reached from $A$ without passing through $C$, and let $E$ be all the rest.
            (Note that $A,B,C,D,E$ are disjoint and, together, account for all the vertices.)
            Argue that for any clique $Q$, either $Q\cap D = \emptyset$ or $Q\cap E = \emptyset$, or both.
        \item Show that $p(x_A,x_B|x_C) = p(x_A|x_C)p(x_B|x_C)$ (i.e., $X_A \perp X_B \mid  X_C$). (Hint: Take the joint distribution, sum over $x_D$ and $x_E$ to get an expression for $p(x_A,x_B,x_C)$, apply (a) to split it into two factors, and then apply exercise~\ref{factor} with $X_A,X_B,X_C$ in place of $X,Y,Z$.)
    \end{enumerate}
\item\label{an} Suppose $p$ respects a DAG $G$. Let $S$ be a subset of vertices, and let $\mathrm{an}(S)$ denote the set of ancestors of $S$ (including $S$).  Show that the marginal distribution $p(x_{\mathrm{an}(S)})$ respects the subgraph of ancestors of $S$ (i.e., the graph obtained by removing any non-ancestors and their edges). (Hint: Take the joint distribution and sum out all non-ancestors.)
\item\label{moral} Show that if $p$ respects a DAG $G$, then $p$ also respects the (undirected) moralization of $G$.
\item In this exercise, you will derive the ``moral ancestral separation criterion'' for conditional independence in directed graphical models.
      Suppose $p$ respects a DAG $G$. 
      Suppose $A,B,C$ are disjoint subsets of vertices of $G$, and let $G_\text{MA}$ denote the moralization of the subgraph of ancestors of $S = A\cup B\cup C$.
      Argue that if all paths in $G_\text{MA}$ from $A$ to $B$ are blocked by $C$, then $p(x_A,x_B|x_C) = p(x_A|x_C)p(x_B|x_C)$ (i.e., $X_A\perp X_B \mid  X_C$).
      (Hint: Use exercises \ref{an} and \ref{moral}, along with the separation criterion for conditional independence in undirected graphical models.)
\end{enumerate}

























\end{document}

























