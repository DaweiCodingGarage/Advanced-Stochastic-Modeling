\documentclass[12pt]{article} 
\input{../../custom}
\usepackage[margin=1in]{geometry}


\begin{document}
\begin{center}
\large\textbf{STA531 Midterm Exam 2 Solutions}
\end{center}

\begin{enumerate}
\item Graphical models

\begin{enumerate}
\item $p(x_1,\ldots,x_8) = p(x_1) p(x_2) p(x_3) p(x_4 | x_1,x_2) p(x_5 | x_2,x_3) p(x_6 | x_4) p(x_7 | x_4,x_5) p(x_8 | x_5)$
\item Yes, Indeterminate, Yes, Indeterminate
\item ~\\
\vspace{-4ex}
%\begin{figure}
\begin{center}
\tikzstyle{hidden}=[circle,
                        thick,
                        draw=black,
                        fill=white]
                        % minimum size=1.5ex,


\tikzstyle{observed}=[circle,
                        thick,
                        draw=black,
                        fill=gray!60]
                        % minimum size=1.5ex,


\tikzstyle{plate}=[rectangle,
                        draw=black,
                        fill=none,
                        inner sep=.8ex,
                        rounded corners=.6ex]

\begin{tikzpicture}[>=latex] %,text height=1.5ex,text depth=0.25ex]
  % The various elements are conveniently placed using a matrix:
  \matrix[row sep=2ex,column sep=1ex] {
        \node (1) [hidden] {$1$};  & &
        \node (2) [hidden] {$2$};  & &
        \node (3) [hidden] {$3$}; \\ &
        \node (4) [hidden] {$4$};  & &
        \node (5) [hidden] {$5$};  & \\
        \node (6) [hidden] {$6$};  & &
        \node (7) [hidden] {$7$};  & &
        \node (8) [hidden] {$8$}; \\
    };
    
    % The diagram elements are now connected through arrows:
    \path[-]
        (1) edge[thick] (4)
        (2) edge[thick] (4)
        (2) edge[thick] (5)
        (3) edge[thick] (5)
        (4) edge[thick] (6)
        (4) edge[thick] (7)
        (5) edge[thick] (7)
        (5) edge[thick] (8)
        (1) edge[thick] (2)
        (2) edge[thick] (3)
        (4) edge[thick] (5)
;
        
\end{tikzpicture}
\end{center}
%\caption{}
%\end{figure}
\end{enumerate}

\item Markov chains and graphical models

\begin{enumerate}
\item Suppose $p$ respect $G_\text{D}$. Then
$$ p(x_1,\ldots,x_n) = p(x_1) \prod_{i = 1}^{n-1} p(x_{i+1} | x_i) = \prod_{i = 1}^{n-1} \varphi_i(x_i,x_{i+1}) $$
if we define $\varphi_1(x_1,x_2) = p(x_1) p(x_2 | x_1)$ and $\varphi_i(x_i,x_{i+1}) = p(x_{i+1} | x_i)$ for $i = 2,\ldots,n-1$. The functions $\varphi_1,\varphi_2,\ldots,\varphi_{n-1}$ are nonnegative, and the maximal cliques of $G_\text{U}$ are $\{1,2 \},\{2,3 \},\ldots,\{n -1,n \}$. Therefore, $p$ respects $G_\text{U}$.
\item Suppose $p$ respects $G_\text{U}$. Then
$$ p(x_1,\ldots,x_n) = p(x_1)\prod_{i = 2}^n p(x_i | x_1,\ldots,x_{i -1}) = p(x_1) \prod_{i = 2}^n p(x_i | x_{i-1}) = \prod_{i = 1}^n p(x_i | x_{\mathrm{pa}(i)})
$$
since
\begin{itemize}
\item $X_i \perp X_1,\ldots,X_{i -2} \mid X_{i -1}$ by the separation criterion for undirected graphical models, because all paths from $i$ to $\{1,\ldots,i -2 \}$ pass through $i -1$, and
\item $\mathrm{pa}(1) = \emptyset$ and $\mathrm{pa}(i) = \{i -1 \}$ for $i = 2,\ldots,n$.
\end{itemize}
Therefore, $p$ respects $G_\text{D}$.
\end{enumerate}

\item Modeling the data collection process

As usual, the appropriate posterior on $\theta$ to use is $p(\theta | y_\text{obs},I)$, since this is the posterior obtained by conditioning on all of the observations.  Since $I_j = 1$ for all $j = 1,\ldots,n$, $\mathrm{obs}= \{1,\ldots,n \}$.  Further, the distribution of $I| y,\theta$ does not depend on $\theta$ (because $p(I | y,\theta) = \prod_{j = 1}^n \phi_{y_j}^{I_j}(1 - \phi_{y_j})^{1 - I_j} \propto_\theta 1$). Consequently, the data collection process plays no role:
\begin{align*}
p(\theta | y_\text{obs},I) &= p(\theta | y,I) \propto p(y,I,\theta) = p(\theta) p(y | \theta) p(I | y,\theta)\\
& \propto_\theta p(\theta) p(y | \theta) \propto_\theta \theta^{a -1}(1 - \theta)^{b -1} \prod_{j = 1}^n \theta^{y_j}(1 - \theta)^{1 - y_j}\\
& \propto_\theta \Beta(\theta \mid a + \textstyle\sum y_j,\, b + n - \sum y_j).
\end{align*}

\item Viterbi algorithm

Define $\nu_n(z_n) = 1$ for each $z_n = 1,\ldots,m$, and denote $p(z_1) = p(z_1 | z_0)$ for notational simplicity. Then
\begin{align*}
\max_{z_{1:n}} p(x_{1:n},z_{1:n}) & = \max_{z_{1:n}} \prod_{t = 1}^n p(z_t | z_{t -1}) p(x_t | z_t)\\
&~~\vdots\\
& = \max_{z_{1:j}} \Big(\prod_{t=1}^{j} p(z_t | z_{t -1}) p(x_t | z_t) \Big) \underbrace{\max_{z_{j +1}} p(z_{j +1} | z_j) p(x_{j +1} | z_{j +1}) \nu_{j +1}(z_{j +1})}_{\textstyle\text{call this } \nu_j(z_j)}\\
& ~~\vdots\\
& = \max_{z_1} p(z_1) p(x_1 | z_1) \nu_1(z_1).
\end{align*}
Therefore, we have the following algorithm:
\begin{enumerate}
\item[(1)] Set $\nu_n(z_n) = 1$ for $z_n = 1,\ldots,m$.
\item[(2)] For each $j = n -1,n -2,\ldots,1$, for each $z_j = 1,\ldots,m$, compute
$$ \nu_j(z_j) = \max_{z_{j +1}} p(z_{j +1} | z_j) p(x_{j +1} | z_{j +1}) \nu_{j +1}(z_{j +1}). $$
\item[(3)] Return $\max_{z_1} p(z_1) p(x_1 | z_1) \nu_1(z_1)$.
\end{enumerate}
(Note: If you showed that your algorithm takes $n m^2$ time, then that is great, but this was not required---as long as your algorithm actually does take $n m^2$ time, then that is sufficient.)
\end{enumerate}

\end{document}






