\documentclass[12pt]{article} 
\input{../custom}

\title{What to know for midterm \#1}
\author{}
\date{}


\begin{document}
\maketitle

\section{BDA 4: Asymptotics and connections to non-Bayesian approaches}

\subsection{Consistency}
\begin{itemize}
\item Know the definition of what it means for an estimator to be consistent.
\item Know the definition of what it means for a posterior distribution to be consistent.
%\item In simple univariate cases with conjugate priors, be able to give a mathematical argument showing that the posterior is consistent when the model is correctly specified. (What I mean here is to show it directly using an explicit formula for the posterior---not using general abstract theorems. Hint: You can do this using the law of large numbers and Chebyshev's inequality.)
\item Understand the concept of frequentist analysis of Bayesian methods, in which we ask what properties a given Bayesian procedure would have if the data were generated from some true distribution $P_0$ (which may or may not be a member of the assumed model class).
\item Know that Doob's theorem provides very general conditions under which posterior consistency holds, if the model is correct specified and identifiable. (You do not need to know the details.)
\item Know that when the model is misspecified, the posterior will typically concentrate at the point $\theta^*$ minimizing the KL divergence. 
\end{itemize}

\subsection{Asymptotic normality}
\begin{itemize}
\item Know what it means that the posterior is asymptotically normal, and be able to write down the formula expressing this. In particular, know the mean and covariance matrix of the normal approximation.
\item Be able to derive the formula for asymptotic normality of the posterior, from the Taylor approximation (without rigorous details).
\item In simple cases with univariate $\theta$, be able to analytically compute the mean and variance of the asymptotic normal approximation, for a given likelihood.
\item (Exercise 4 from homework 1) Understand why the posterior on $\phi$ is (typically) asymptotically normal when $\phi = f(\theta)$, and know how the mean and variance of the asymptotic normal distribution change under this transformation.
\item Be able to give an argument for why asymptotic normality of the posterior, plus consistency of the MLE, typically will imply posterior consistency.
\item Understand some of the ways in which posterior consistency and asymptotic normality can fail, and be able to give examples.
\end{itemize}

\subsection{Frequentist coverage}
\begin{itemize}
\item Know the definition of the coverage probability of a confidence region, and have a good intuitive understanding of what it means.
\item Understand why having good frequentist coverage is a desirable property.
\item In simple cases, be able to analytically compute coverage probability.
\item Understand the definition of a posterior credible region (e.g., a  90\% or 95\% credible region).
\item Know the definition of an equal-tailed posterior credible interval.
\item In simple cases, be able to analytically compute an equal-tailed posterior credible interval.
\item Know that posterior credible regions often (but not always) have good frequentist coverage properties.
\item Understand why, if the prior and likelihood are exactly correct, posterior credible regions have frequentist coverage equal to their posterior probability (exercise 15a from homework 1).
\end{itemize}


\section{BDA 6 \& 7: Model checking and cross-validation}

\subsection{Posterior predictive checking}
\begin{itemize}
\item Understand the idea behind posterior predictive checks.
\item Know how to perform a posterior predictive check and compute a posterior predictive p-value.
\item In simple cases, be able to analytically compute a posterior predictive p-value.
\item Know the definition of the posterior predictive distribution for replicate data sets, and know how to sample from it based on posterior samples.
\item Know how to interpret the results of a posterior predictive check. 
\item Understand that, ideally, p-values are uniformly distributed, so sometimes we will see p-values close to zero or one simply by chance. If we were to compute a large number of p-values, know how many we would expect to see outside a given range.
\item Know that (unfortunately) posterior predictive p-values are not ``true'' p-values in the sense that they are not uniformly distributed, even if the model is correct.
\item Understand why some test statistics/quantities will always be well captured by a given model (and thus are not very informative about model fit), especially in the case of exponential families.
\item Realize that one needs to be careful when modifying the model based on the results of posterior predictive checks, since this can lead to overfitting.
\item Realize that posterior predictive checks represent a sort of internal consistency check, but that they are not an ideal way of evaluating model fit, because they are ``using the data twice''.
\end{itemize}

\subsection{Cross-validation}
\begin{itemize}
\item Understand the idea behind cross-validation---why does it make sense?
\item Know the definition of leave-one-out cross-validation.
\item Know the definition of $k$-fold cross-validation.
\item Know these common choices of loss function for cross-validation: log posterior predictive, 0-1 loss, square loss.
\item Be able to derive the expected loss for a given loss function (taking care to compute it with respect to the true distribution!)
\item Know how to compute a cross-validation estimate of generalization performance.
\item Understand why cross-validation will typically provide a better assessment of performance compared to posterior predictive checks, since CV is not evaluating the model on the same data that was used to fit the model.
\item Understand that if cross-validation is used to choose among multiple models, then in order to assess the performance of the chosen model, it needs to be evaluated on a further held-out set (disjoint from the set of data used for cross-validation).
\end{itemize}


\section{BDA 8: Modeling accounting for data collection}

\begin{itemize}
\item Be able to recognize situations in which the data collection process is biased in a way that will affect your inferences.
\item Be able to give specific examples of situations in which it is important to model the data collection process.
\item Know the definition of ignorability, as well as the intuitive interpretation of it.
\item Be able to explain the interpretation of the ``potential outcomes'' $y$ and the observation indicators $I$.
\item Know the definition of the ``complete-data likelihood'' in the general setup we considered.
\item Know what distribution to use for posterior inferences about $\theta$ when ignorability does not hold.
\item Be able to derive a formula for this posterior in simple cases.
\item Given a verbal description of a distribution on potential outcomes and a data collection process, be able to write down a reasonably appropriate probabilistic model for it.
\item Know the definitions of the following conditions: missing at random (MAR), missing completely at random (MCAR), strong ignorability, and distinct parameters.
\item Be able to give examples in which these conditions hold or do not hold.
\item Know what implications hold between these different conditions.
\item Be able to prove that strong ignorability implies ignorability.
\item Be able to prove that MAR + distinct parameters implies ignorability.
\end{itemize}


  


\end{document}






